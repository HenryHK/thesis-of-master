\title{Unforkable blockchain cryptocurrencies with efficient zero knowledge contingent proof on mobile devices}
%----------------------------------------------------------------------------------------
%	PACKAGES AND OTHER DOCUMENT CONFIGURATIONS
%----------------------------------------------------------------------------------------

\documentclass[12pt]{article}
\usepackage[english]{babel}
\usepackage[utf8x]{inputenc}
\usepackage{amsmath}
\usepackage{graphicx}
\usepackage[colorinlistoftodos]{todonotes}
\usepackage{float}
\usepackage{csquotes}

\begin{document}

\begin{titlepage}

\newcommand{\HRule}{\rule{\linewidth}{0.5mm}} % Defines a new command for the horizontal lines, change thickness here

\center % Center everything on the page
 
%----------------------------------------------------------------------------------------
%	HEADING SECTIONS
%----------------------------------------------------------------------------------------

\textsc{\LARGE The University of Sydney}\\[1.5cm] % Name of your university/college
\textsc{\Large School of Information Technology}\\[0.5cm] % Major heading such as course name
\textsc{\large INFO5993 Research Methods - Assignment II}\\[0.5cm] % Minor heading such as course title

%----------------------------------------------------------------------------------------
%	TITLE SECTION
%----------------------------------------------------------------------------------------

\HRule \\[0.6cm]
{ \huge \bfseries Unforkable Blockchain Cryptocurrencies with Efficient Zero Knowledge Contingent Proof on Mobile Devices}\\[0.4cm] % Title of your document
\HRule \\[1.6cm]
 
%----------------------------------------------------------------------------------------
%	AUTHOR SECTION
%----------------------------------------------------------------------------------------

\begin{minipage}{0.4\textwidth}
\begin{flushleft} \large
\emph{Author:}\\
Lin \textsc{Han} % Your name
\end{flushleft}
\end{minipage}
~
\begin{minipage}{0.4\textwidth}
\begin{flushright} \large
\emph{Supervisor:} \\
Dr. Vincent \textsc{Gramoli} % Supervisor's Name
\end{flushright}
\end{minipage}\\[2cm]

%----------------------------------------------------------------------------------------
%	DATE SECTION
%----------------------------------------------------------------------------------------

{\large \today}\\[2cm] % Date, change the \today to a set date if you want to be precise

%----------------------------------------------------------------------------------------
%	LOGO SECTION
%----------------------------------------------------------------------------------------

\includegraphics{logo.png}\\[1cm] % Include a department/university logo - this will require the graphicx package
 
%----------------------------------------------------------------------------------------

\vfill % Fill the rest of the page with whitespace

\end{titlepage}

\tableofcontents
\vfill

\newpage

% \begin{abstract}
% Your abstract.
% \end{abstract}

\section{Introduction}

From the time when Block 0 of the Bitcoin blockchain, the Genesis Block, is created at 18:15:05 GMT on January 3rd, 2009, the words ``cryptocurrencies'' and ``Blockchain'' become one of the most popular fields in information technology. The ``decentralized'' and ``anonymous'' nature of cryptocurrencies overcomes the weakness of traditional \textit{trust-based} electronic payments who relies heavily on trusted third-party financial institutions. The \textit{cryptographic proof-of-work} of bitcoin enables reliable transaction between  two parties directly\cite{nakamoto2008bitcoin}. Through almost 10 years development, cryptocurrencies turns out to be a large family and can be deployed onto multiple devices. 

Though bitcoin give a great solution on \textit{double spending} problem, it doesn't mean that it is secured in all aspects. One possible issue is attacks targeting blockchain's forkable feature. Other cryptocurrencies allowing forkable chains all suffer from the very same issue. In this sense, unfokable blockchain is proven to solve this problem\cite{DBLP:journals/corr/CrainGLR17}. On the other hand, another possible solution to secure transaction is to adopt mechanism like \textit{zero knowledge contingent payment} which are released if and only if some knowledge is disclosed by the payee and to do this in a trustless manner where neither the payer or payee can cheat\cite{wiki2011zero}\cite{sasson2014zerocash}. Whilst there are several theoretical discussion and practice in a variety of contexts, this paper will concentrate on their application on cryptocurrencies blockchain, especially on mobile devices. 

\section{Bitcoin}
\label{sec:Blockchain}

\subsection{Blockchain}

Blockchain is the way how Bitcoin keeps its public ledger. To some extent, blockchain is simply a peer-to-peer distributed timestamp server. The ultimate goal of this design is to solve \textit{double-spending} problems and prevent modification of transaction records\cite{nakamoto2008bitcoin}. 

Each full node in the Bitcoin network keeps a full copy of the blockchain, in which all blocks validated by this particular is stored. When several nodes within the network independently arrive at identical blockchains, they are considered to be in \textit{consensus}. As its name suggests, a blockchain is a digital chain of blocks, where a timestamp, a nonce, and a Merkle Tree is stored.  The blocks are chained cryptographically using hash. In detail, each block contains the hash of its previous block ,finally leading to the Genesis Block. Any modification on blocks in the chain would violates all subsequent hashes, which is vital for consistency of the ledger. Figure \ref{fig:blockchain} shows part of a blockchain. 

% blockchain figure here
\begin{figure}
    \includegraphics{blockchain.png}
    \caption{Blockchain}
    \label{fig:blockchain}
\end{figure}

However, computing a hash is expensive. This truth enables the adoption of \textit{proof-of-work} in bitcoin network. 

\subsection{Proof of Work}

According to blockchains' feature, a huge amount of computation is required in the generation of each block. Meanwhile, there is a \textit{proof-of-work} mechanism to make the distributed timestamp server work and determine representation in majority decision making. Especially, when there are multiple chains (forked chains), consensus rules will pick up the longest chain, which contains the most proof of work during its generation\cite{nakamoto2008bitcoin}.

In this way, any malicious changes on previous blocks would violate its following blocks. That is to say, hacker with huge computing power can hijack the blockchain if he can generate the longest chain from the block he hacks, in turn, he has to own more than half of the computing power within the whole blockchain network\cite{NG17}.

\subsection{Contracts}

There are distributed contracts in Bitcoin transactions for agreement enforcements, which provides another way to formalize and guarantee agreements rather than traditional court system. Examples include Escrow, Micropayment channels and CoinJoin.

Some of the contracts can be implemented in Bitcoin Script, especially the zero knowledge contingent payments in Bitcoin is achieved using it. However the Red Belly Blockchain doesn't have a robust script language like Bitcoin does.

\section{Ethereum}

\subsection{Previous Work}

Bitcoin provides a protocol allowing weak implementation of \textit{smart contracts}. However, several limitations exists in Bitcoin's scripting language:

\begin{enumerate}
    \item \textbf{Not Turing-Completeness} - Bitcoin scripts lacks loops.
    \item \textbf{Lack of States} - UTXOs scripts is only for one-off contracts.
    \item \textbf{Blindness of Blockchain} - Bitcoin scripts cannot access blockchain data.
    \item \textbf{Blindness of Value} - Bitcoin either consumes the entire UTXO or none of it
\end{enumerate}

\subsection{Rationale}

Ethereum implements a blockchain with Turing-complete scripts, states, value awareness and blockchain awareness, which enables development of smart contracts, and even new protocols\cite{wood2014ethereum}.

\section{Balance Attack}
\label{sec:Balance Attack}

As the previous review mentioning, to attack a blockchain, or specifically to rewrite the content of a block, the hacker should have more than half of the computing power of the whole blockchain network which is almost unfeasible in real world. In particular, by delaying the propagation of blocks in Bitcoin system, the hacker can in result delay the growth of the longest branch of the system. In other word, he can then hijack the blockchain even without a large amount of computing power. Ethereums' ``Blockchain 2.0'' somehow fixes this problem, but there is still other possible method against forked blockchain. One practice is the \textbf{Balance Attack} \cite{Gra16}\cite{NG17}\cite{NG16}against \textit{proof-of-work} blockchain systems.

To achieve a balance attack within the blockchain network, the attacker should divide the network into subgroups of similar mining power by cutting off their communications. During this down time, the attacker issues transaction in the \textit{transaction group}, and mine blocks in the \textit{block group} simultaneously. This action only ends when it comes to the point where the tree of the block subgroup outweighs the tree of the transaction group, which is with high possibility. The balance, in result, can leverage the \textit{GHOST} protocol that accounts for sibling or uncle blocks to determine on a chain of blocks. This strategy allows the attacker to mine a branch regardless of the rest of the network so that he can influence the branch determination process while merging\cite{NG17}. The process is as shown in Figure \ref{fig:balance_attack}.

\begin{figure}
    \includegraphics{balance_attack.png}
    \caption{Balance Attack}
    \label{fig:balance_attack}
\end{figure}

\section{Unforkable Blockchain}
\label{sec:Unforkable Blockchain}

\subsection{Byzantine Consensus Problem}

The \textit{Byzantine Consensus Problem} refers to the \textit{Byzantine General's Problem} proposed by Leslie Lamport, Robert Shostak and Marshall Pease in 1982. The problem is complicated by the presence of traitorous generals who may not only cast a vote for a suboptimal strategy, they may do so selectively. All the votes and results are simplified to attack or retreat. The problem is complicated further by the generals being physically separated and having to send their votes via messengers who may fail to deliver votes or may forge false votes\cite{lamport1982byzantine}.

In computer science area, typically computers or participants in network are mapped to generals and links between them are mapped to messengers.

\subsubsection{Traditional Blockchain Byzantine Consensus Problem}

The \textit{proof-of-work} of Bitcoin blockchain is the primary solution to \textit{Byzantine Consensus Problem}. 

In detail, a model of Bitcoin network can be built upon the classic Byzantine Consensus Problem. The distributed system is the alliance of generals, in which the upper bounds on the delay of communicating and decision-making is unknown. 

However, regarding to our previous discussion on \textit{Balance Attack}, attackers can still disrupt the consensus system to beat the Bitcoin Byzantine Consensus\cite{gramoli2017blockchain}. Because there is no guarantee that the decided value is proposed by a valid process.

\subsubsection{Democratic Byzantine Consensus}

Democratic Byzantine Fault Tolerance is a system specially tailored for consortium blockchains based on \textit{Binary Byzantine Consensus}. In \textit{Binary Byzantine Consensus}\cite{mostefaoui2015signature}, each trusted participant issue proposal with either 0 or 1 and decides that the final agreement such that:

\begin{enumerate}
    \item No pair of trusted participants have different decision.
    \item Every trusted participant decides
    \item If all correct participant propose the the same value, then no other value can be deiced
\end{enumerate}

In safe Democratic Byzantine Fault Tolerance\cite{DBLP:journals/corr/CrainGLR17}, a mechanism called binary broadcast for binary Byzantine Consensus system is adopted. To conclude, there are four aspects that is strictly followed within DBFT:

\begin{enumerate}
    \item \textbf{Binary Value Obligation} - if t+1 correct BV-broadcats $v$, then $v$ is eventually added to the set binary values of all correct process.
    \item \textbf{Binary Value Justification} - if $p_i$ is correct and and $v$ in $binary-values_i$ then $v$ was broadcasted by a correct process.
    \item \textbf{Binary Value Uniformity} - if $v$ is added to $binary-value_i$ of correct $p_i$, then eventually $v$ will be in $binary-value_j$ for all correct $p_j$.
    \item \textbf{Binary Value Termination} - eventually $binary-value$ of correct $p_i$ is not empty. 
\end{enumerate}

\subsection{The Red Belly Blockchain}

Red Belly Blockchain is new blockchain relying on the Democratic BFT. The \textit{Genesis Block} of this blockchain contains the initial information as well as a list of n participants. All accesses of external nodes requires these participants as the middleware. In this case, a transaction is regarded as committed if $t+1$ participating nodes agreed so it can be written into next block.

The performance of this new blockchain is quite good. It can achieve more than 400 transactions per second and with great scalability up to 90 server nodes.

\section{Zero Knowledge Contingent Payment}
\label{sec:Zero Knowledge Contingent Payment}

\subsection{Bitcoin}

Zero Knowledge Contingent Payment in Bitcoin is first proposed by Gregory Maxwell in 2011 on Bitcoin Wiki\cite{maxwell5zero}. The basic process can be illustrated by an example\cite{campanelli2017zero}:

\begin{displayquote}
    Alice is a fan of Sudoku puzzle. However, there is a puzzle that she is trying days but in vain. She gives up and broadcasts a message within a fan group proclaiming that ``I will pay whoever solves this puzzle.'' Bob see the broadcast, solves it, and want to sell the result to Alice. The problem is either Alice or Bob is willing to be the first person to give out what they have.

    To solve this dilemma, Alice and Bob goes for a Bitcoin, which allows one to issue a transaction and also specify the conditions to be met in order to claim the transaction. In this case, Alice propose a payment transaction to blockchain that includes encoded Sudoku puzzle and the rules. Whoever solves the puzzle is able to get the fund.

    Using the Bitcoin ZKCP protocol, Bob knows a solution $s$ and encrypts the solution using a key $k$ such that $Enc_k(s)=c$ and he computes $y$ such that $SHA256(k)=y$. He then send the key $k$ and $c$ to Alice together with a zero knowledge proof that $c$ is an encryption of $s$ under the key $k$ and  that $SHA256(k)=y$. Once Alice has verified the proof, she creates a transaction to the blockchain that pays Bob $n$ bitcoins, and says that Bob can only claim the coins if he provides the value $k'$ such that $SHA256(k')=y$. Bob then published $k$ and claims the fund. In this way, Alice learns $k$ can decrypt c so that she knows the solution $s$
\end{displayquote}

Specifically in Blockchain, the ZKCP protocol now has several practices rather than theory. One is ZK-SNARK protocols allowing for the practical implementation of the necessary proofs\cite{kalai2006succinct}\cite{ben2015secure}\cite{ecc2011}.

\subsection{Zero Knowledge Contingent Payment without Scripts}

However, all implemented ZKCP protocols now are based on scripting language in traditional forkable blockchain\cite{Banasik2016}. Besides, there is no convincing data of its performance on modern mobile devices\cite{doi:10.1080/00207160.2014.933816}.

\section{Conclusion}

In this literature review, we can see that unforkable blockchain has a more reliable security guarantee compared to traditional forkable blockchain used in mainstream cryptocurrencies. The need to deploy unforkable blockchain and implement ZKCP-enable protocols onto mobile devices is evident through the literatures to fulfill the gaps between the theoretical discussion and real life application. Deploying ZKCP-enbale unforkable blockchains onto mobile devices will benefit both authorities seeking reliable blockchain solutions and users having trust problems with third-parties.

\newpage
\bibliography{bibliography.bib}
\bibliographystyle{plain}

\end{document}
